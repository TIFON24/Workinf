\documentclass[russian,utf8]{eskdtext}
\usepackage[T2A]{fontenc}
\usepackage[utf8]{inputenc}
\usepackage{amsmath,amssymb,amsthm}
\usepackage[english,russian]{babel}
\usepackage{tikz}
\usepackage{siunitx}
\usepackage[american,cuteinductors,smartlabels]{circuitikz}
\renewcommand{\baselinestretch}{1.5}


\ESKDdepartment{Федеральное агентство по образованию}
\ESKDcompany{Санкт-Петербургский государственный электротехнический университет "ЛЭТИ"}
\ESKDtitle{Пояснительная записка к Курсовой работе}
\ESKDsignature{Вариант N29}
\ESKDauthor{Штук А.К.}
\ESKDchecker{Прокшин А.Н.}
\ESKDdocName{по дисциплине "Информатика"}

\begin{document}

\maketitle
\tableofcontents


\newpage
\section{Цель и тема курсовой работы}
 \textbf{Цель курсовой работы:} уметь применять персональный компьютер и
математические пакеты прикладных программ в инженерной деятельности.
 
 \textbf{Тема курсовой работы:} решение математических задач с использованием
математического пакета "Scilab"или "Reduce-algebra".

\newpage
\section{Иследование функции}

\subsection{Решить уравнение}

\subsection{Исследовать функцию}

\newpage
\section{Иследование кубического сплайна}

\newpage
\section{Задача оптималного распределения неоднородных ресурсов}

\newpage
\section{Заключение}

\newpage
\section{Список литературы}

\end{document}